\documentclass[11pt,a4paper]{jarticle}
\usepackage[dvips]{graphicx}
\usepackage{here}

\title{{チーム研究\\実験計画}}
\date{2014年6月3日(火)}
\author{小森谷 大介、川畑 裕也、深津 佳智}

\begin{document}

\maketitle

%1
\section{概要}
昨年度の研究から得た知見を基に、表面及び背面に突起もしくはくぼみを付けた新たなケースのデザインを考える。
具体的には、スマートフォンの表面(タッチパネル)に突起を付けることにより、ユーザがタッチパネルにタッチする際の手がかりとするという発展案を考えた。
作成したプロトタイプ、及び、実験計画について報告する。

%2
\section{突起の位置の選定}
小森谷セクション

%3
\section{プロトタイプ}
川畑セクション

\section{実験計画}
作成したプロトタイプを用いて、昨年度のソフトウェア科学会大会投稿時の実験と同様の実験を行うことを考えている。
具体的には、
\begin{itemize}
	\item 突起条件:突起無し条件、表面突起A地点条件、表面突起B条件、表面突起C地点、背面突起A地点条件、背面突起B条件、背面突起C地点
	\item 分割条件:3~$\times$~3分割条件、4~$\times$~4分割条件、5~$\times$~5分割条件
\end{itemize}
の2種類の実験条件を設定し、被験者にアイズフリーにおいてタッチパネル上のターゲットをタッチしてもらう実験を行うことを考えている。
昨年度の実験と今回の実験計画の相違点は、ケース条件であったところを突起条件に変え、表面及び背面の両方において実験を行う点である。
また、昨年度の実験においては開始点タッチ条件と終了点タッチ条件が存在したが、
表面に付いている突起をタッチの手がかりとできることを考え、終了点タッチ条件のみにした。(昨年度の実験では、開始点タッチ条件と終了点タッチ条件のタッチ精度に有意差は見られなかった)。

%4
\section{サーベイ}
どうしようか?
一人一本いけます?

%5
\section{今後の予定}
7月に研究会かHISに出し、HCIIに出せるのが理想的なルート
%研究会とかHCIIとか書いておけば良いかと
\begin{itemize}
  \item 6月
  \begin{itemize}
    \item 実験
  \end{itemize}
  \item 7月
  \begin{itemize}
    \item HCI研究会(日程未定)
  \end{itemize}
  \item 8月
  \begin{itemize}
    \item 31日 HCIIアブスト締切(去年の日程)
  \end{itemize}
\end{itemize}
\bibliographystyle{jalpha}
\bibliography{shokunyu}
\end{document}
